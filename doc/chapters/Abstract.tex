\begin{abstract}
	This report introduces CogniPredictAD, a \textit{Data Mining and Machine Learning} project designed to analyze clinical and biological parameters from the ADNI dataset, with the goal of predicting final clinical diagnoses (CN, EMCI, LMCI, AD) based on baseline data.
	
	The analysis includes a detailed examination of preprocessing techniques: missing data management, normalizations, and feature engineering. We assess the models based on their Macro F1 Score and present the results through various evaluations.
	
	The modeling phase includes Decision Tree, Random Forest, Extra Trees, Adaptive Boosting and Multinomial Logistic Regression (with and without sampling on the dataset). Hybrid Sampling techniques, Grid Search for hyperparameter tuning, and cross-validation are applied. In the end the best performance is obtained by Extra Trees (without Hybrid Sampling) with Accuracy $\approx$ 0.9442, Macro F1 Score $\approx$ 0.9376, and ROC-AUC $\approx$ 0.9867.
	
	We decided, for improving explainability, to also keep Decision Tree (it proved better with Hybrid Sampling) which has Accuracy $\approx$ 0.9153, Macro F1 Score $\approx$ 0.9055, and ROC-AUC $\approx$ 0.9773.
	
	Due to the ambiguity of the high predictivity of three attributes (CDRSB, LDELTOTAL, mPACCdigit), I built two alternative models with the same pipeline as the main ones, but without those features.  
	
	The conclusions indicate strong predictive performance on the ADNIMERGE dataset but emphasize that potential sample bias and the absence of external validation limit clinical applicability. Additional patients to the dataset or through data integration with comparable datasets are required before any clinical use.
\end{abstract}