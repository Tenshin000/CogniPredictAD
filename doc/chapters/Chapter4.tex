\section{Data Exploration}
Explorations revealed that the dataset has 16,421 rows and 116 columns. However, these records represent the various visits, and I was only interested in the baseline ones. The dataset contains 2,419 useful patients (using "useful" means those who did not have a NULL baseline diagnosis) for the proposed problem.

Many columns contain significant percentages of missing cases. The diagnostic classes of the baseline sample are unbalanced, but not extremely unbalanced.

Demographic and risk analyses show bias in the ADNI sample. Ethnicity is heavily skewed toward white subjects, with high average levels of education, and many married individuals. There are more men than women, but overall the number is not disproportionate. All this, however, implies that models may perform worse on more heterogeneous clinical populations. 

The \textit{Data Exploration} was then divided into three parts: 
\begin{enumerate}
	\item the \textit{preliminary data exploration} of the raw dataset;
	\item the \textit{data exploration} after splitting and preparing the actual learning set;
	\item the \textit{data exploration} after Preprocessing to select the classification models. 
\end{enumerate}

The \textit{preliminary data exploration}'s descriptive summaries revealed extensive missing data, especially in key biomarker and imaging variables, prompting a structured missing-data plan. Review of baseline and visit fields confirmed usable \textit{DX\_bl} labels and clarified visit patterns, motivating a restriction to baseline records for classification.

In the \textit{data exploration}, the training set was inspected using a reduced set of clinically relevant variables and derived ratios. Summary statistics, boxplots, and correlation maps were used to assess distributions and multicollinearity. Clinical and biomarker distributions showed the expected disease gradient but with strong overlap, skewness, and outliers. Sex differences in MRI volumes confirmed the need for ICV normalization. Class counts showed \textit{only mild imbalance} across CN, EMCI, LMCI, and AD. 

After preprocessing on the training set, I performed a \textit{new data exploration}. The dataset included demographics, APOE4, cognition, Ecog, CSF ratios, and MRI/ICV volumes, with feature distributions summarised using medians and IQRs.

Outlier Detection was carried out using the \textbf{Local Outlier Factor} on robustly normalised data, applied to clinical, MRI/ICV, CSF, and full-feature groups. Only a small set of subjects showed high anomaly scores: one cognitive outlier, two MRI/ICV outliers, and five CSF biomarker outliers. Most represented biologically plausible extreme profiles (e.g., severe AD cases or unusual biomarker combinations), indicating that anomalies reflect genuine variability rather than errors. This supports adopting models tolerant to skewness and extreme values rather than removing these points.

Group differences across diagnoses were evaluated with the \textbf{Kruskal--Wallis test}, chosen because many features showed skewed, non-Gaussian distributions with outliers. Almost all cognitive, functional, CSF, and MRI measures showed highly significant differences across CN, EMCI, LMCI, and AD, while demographic variables displayed weaker effects. These results confirm that the main clinical and biological features vary systematically across disease stages and are informative for downstream classification.



